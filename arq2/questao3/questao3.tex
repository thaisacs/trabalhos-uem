\documentclass[12pt]{article}

\usepackage{sbc-template}

\usepackage{graphicx,url}

\usepackage[brazil]{babel}   
\usepackage[utf8]{inputenc}

     
\sloppy

\title{Previsão de desvio dinâmica}

\author{Rafael Rodrigues dos Santos\inst{1}, Thais Aparecida Silva Camacho\inst{1}}


\address{Departamento de Informática -- Universidade Estadual de Maringá
  (UEM)
  \email{rafael11rodrigues@hotmail.com, thaiscamachoo@gmail.com}
}


\usepackage{cite}

\begin{document}
\maketitle

\section{Introdução}

Um dos maiores desafios no projeto de processadores que suportam a execução de múltiplas instruções simultaneamente é minimizar o tempo em que partes do
processador não são usadas, visto que instruções dependentes não podem ser executadas em paralelo, o que faz essas instruções ficarem paradas até que as
dependências sejam resolvidas.

Um dos tipos de dependência bastante problemático é o de dependência procedural, que diz respeito às instruções de desvio. Quando existe um salto condicional, a
execução do programa pode ou não ser desviada para um outro ponto, quebrando o fluxo sequencial. Se o salto ocorre, as instruções imediatamente subsequentes à
instrução de desvio não devem ser executadas. Todavia, é apenas na execução que se descobre se o salto deve ou não ser tomado e, até chegar nesse ponto,
algumas instruções podem ter sido executadas indevidamente e o processador precisa retomar ao ponto correto da execução.

Uma vez que essa retomada custa caro em termos de tempo e processamento, é preciso adotar medidas que possam deduzir se um desvio será tomado ou não, pois com
essa informação é possível impedir que instruções indevidas sejam executadas. Tais medidas são denominadas de previsão de desvio e são divididas em previsão
estática e dinâmica, cada categoria tendo suas vantagens e desvantagens. Este documento abordará os algoritmos de previsão dinâmica, trazendo algumas características
 dos algoritmos mais utilizados e comparando a eficiência de cada um, além de apresentar alguns aspectos da implementação de previsão de desvio no processador
 UltraSPARC I.

\section{Fundamentação Teórica}

Quando existe uma instrução de desvio em um programa, duas operações devem ser feitas: o cálculo do endereço da instrução para a qual o programa será desviado e
a avaliação dos operadores que determinam se o salto será tomado ou não. Ambas ocorrem apenas na execução, isto é, por mais que a etapa da decodificação
identifique que se trata de um salto condicional, ainda não é possível dizer se o salto deve ser tomado.

  Enquanto não se sabe o que deve ser feito, duas situações podem ocorrer com relação às instruções subsequentes: elas podem gastar alguns ciclos de clock
  esperando a decisão do desvio\cite{schemes} ou elas podem ser buscadas e executadas. No segundo caso, se for definido que o desvio deve ser tomado, as
  instruções subsequentes não deveriam ter sido executadas e o processador gasta ainda mais ciclos limpando essas instruções indevidas\cite{smith}. De qualquer
  forma, há uma queda no grau de paralelismo e principalmente de desempenho do processador.
  
  Os algoritmos de previsão de desvio tentam contornar esse problema, ou seja, a cada instrução de desvio identificada eles tentam direcionar a execução do
  programa para as instruções com maiores chances de realmente serem executadas posteriormente. Tais algoritmos podem ser estáticos ou dinâmicos, sendo que os
  dinâmicos apresentam maiores taxas de acerto\cite{schemes} por utilizarem dados da execução de cada programa em vez de definir uma dada política antes da
  execução.
  
  As estratégias de previsão de desvio dinâmica são subdivididas em duas categorias, as quais serão brevemente explicadas a seguir. 
  
  \subsection{Previsão de desvio de 1 nível}
  
  Esse tipo de previsão de desvio se baseia em apenas uma informação do histórico de previsão: se os desvios foram tomados ou não. Geralmente são implementados
  por contadores ou por uma tabela de previsão de desvios.
  
  Um contador é incrementado quando um desvio é tomado e decrementado caso contrário. A forma mais simples é usar um contador de 1 bit, que prevê que o desvio
  não será tomado quando está em 0 e que será tomado quando está em 1. Contudo, essa estratégia erra em dois momentos: na primeira iteração de um laço de
  repetição (usualmente ao menos uma iteração é feita) e na última iteração (o desvio para uma nova iteração certamente não é tomado).
  
  A implementação mais comum é a de contadores de dois bits, que assume valores de 0 a 3. O bit mais significativo é o responsável pela predição (análogo ao
  contador de 1 bit). Assim, se em um dado momento o contador está em 11 (previsão de que o desvio será tomado), é preciso errar duas vezes para que o
  processador passe a prever que o desvio não será tomado. Desta forma, por mais que a previsão erre na última iteração de um laço, a previsão na primeira
  iteração do próximo laço certamente será correta.
  
  A tabela (ou buffer) de previsão de desvio -- também conhecida como previsão bimodal -- é uma estrutura que associa uma previsão de desvio para cada instrução
  de desvio executada. Isso é feito mapeando alguns dos bits menos significativos do endereço das instruções de desvio para um conjunto de bits de
  previsão (normalmente contadores de dois bits).
  
  Um cuidado que se deve ter é com relação ao número de bits usados no mapeamento, pois eles determinam o tamanho da tabela. Tabelas muito grandes podem ocupar
  espaços desnecessários, mas tabelas muito pequenas permitem que duas instruções possam ser mapeadas para um mesmo conjunto de bits de previsão, fenômeno
  conhecido como aliasing e que nem sempre tem efeito negativo\cite{benlee}.
  
  \subsection{Previsão de desvio de 2 níveis}
  
  Em vez de verificar apenas se um dado desvio foi tomado ou não anteriormente, a previsão de dois níveis analisa padrões de desvio presentes no programa em
  execução e os armazena em uma tabela de histórico de previsão. Várias foram as propostas de implementação, dentre elas o histórico local, histórico global,
  previsão global com seleção de índice e previsão global com índice compartilhado. Todas essas formas de previsão são chamadas de previsão
  correlacionada\cite{schemes}, pois se baseiam no comportamento conjunto das n instruções mais recentes.
  
  O histórico local considera duas tabelas. A primeira mapeia alguns dos bits menos significativos dos endereços das instruções de desvio para um conjunto de n
  bits que representam o padrão das n execuções mais recentes da instrução. Cada padrão é mapeado em uma segunda tabela, a qual indica se o desvio deve ser
  tomado ou não quando um padrão de desvios se repetir. Esta segunda tabela, chamada de tabela local, pode utilizar apenas 1 bit para indicar a
  decisão\cite{yehpatt} ou um contador de dois bits\cite{smith}, análogo à previsão bimodal.
  
  O histórico global é semelhante ao histórico local, porém substitui a primeira tabela por um registrador de deslocamento de n bits que armazena o resultado
  das últimas n instruções de desvio sem distinção. Isso permite prever desvios que dependam de outros desvios, já que os padrões são
  formados por todas as instruções de desvio do programa\cite{mcfarling}. Por outro lado, essa estratégia não permite identificar sobre qual instrução a
  previsão está sendo feita.
  
 Para resolver a perda de identidade de instruções trazida pelo histórico global, é possível usar a previsão global por seleção (gselect) ou por
 compartilhamento (gshare).
 Ambas as técnicas combinam os bits do histórico global com os bits de endereço da instrução de desvio. A primeira concatena alguns dos bits menos significativos de
 ambos os conjuntos de bits, e a segunda realiza uma operação ``ou exclusiva'' (XOR) bit a bit entre os conjuntos. A vantagem da gshare sobre a gselect é que
 nesta última apenas alguns dos bits de endereço da instrução são usadas, sendo mais difícil identificar a instrução\cite{schemes}, visto que pode haver outras
 instruções e outros padrões de histórico global que formem a mesma sequência de bits.

\subsection{Previsão combinada}

Uma vez que cada técnica de previsão de desvio tira vantagem de observações diferentes da execução dos programas, é possível criar novas formas de previsão
combinando duas técnicas distintas. Isso pode ser feito por meio de uma tabela de contadores de dois bits que, para cada entrada (parte de endereços de
instruções de desvios ou bits de padrões), o contador correspondente seleciona qual das duas técnicas deve ser usada \cite{mcfarling}. O valor do contador depende dos acertos
e erros das duas técnicas. Se ambas as previsões erram ou acertam, o valor não é alterado; se apenas a segunda acerta, o contador é decrementado; e se apenas a
primeira acerta, o contador é incrementado.

Uma boa combinação de previsores de desvio é usar a técnica bimodal e a gshare \cite{mcfarling}, com a qual foi possível aumentar em até três vezes a taxa de
acerto de previsão nos testes feitos por McFarling utilizando o conjunto de programas SPEC'89 Benchmarks.

\section{Implementação}
  
  O UltraSPARC I é um processador da família UltraSPARC desenvolvida pela Sun Microsystems iniciada em 1994. Utiliza-se de uma arquitetura RISC
  superescalar com pipeline reduzido (miniestágio).
  
  A técnica de previsão de desvio empregada pelo UltraSPARC é a bimodal, utilizando dois bits a cada duas instruções de desvio na memória cache de instruções
  (I-cache).
  A tabela de previsão de desvio guarda até 2048 entradas (contadores de 2 bits), o que é bem mais que o necessário pela maioria dos programas. Os contadores
  das instruções de desvio da I-cache são inicializados estaticamente com 01 ou 10 e, conforme elas são executadas, os bits são alterados de acordo com o que foi explicado anteriormente na previsão de 1 nível.
  
  \section{Conclusão}
  
  Independente do grau de paralelismo oferecido por um processador superescalar, é impossível utilizar toda sua capacidade o tempo todo, pois além de alguns
  ciclos de clock serem gastos esperando que dependências de dados sejam resolvidas, saltos condicionais também reduzem o seu desempenho por criarem dois
  possíveis fluxos de execução, sendo que um deles é executado indevidamente. As técnicas de previsão de desvio tentam garantir que apenas o fluxo correto seja
  executado. Existem diversas formas de fazer essa previsão, cada uma com suas vantagens e desvantagens. 
  
  Técnicas de previsão dinâmica são mais eficientes que técnicas de previsão estática, já que utilizam informações obtidas conforme as instruções vão sendo
  executadas em vez de predefinir um comportamento padrão. Um histórico das instruções de desvio é armazenado e, com base nele,o comportamento das próximas
  ocorrências desses desvios é previsto. Entende-se por próximas ocorrências a repetição de uma instrução específica ou de um padrão de desvios atrelado a um
  conjunto de instruções.
  
  Antes de começar a organizar o histórico de desvios -- o qual pode ser implementado de várias maneiras -- é preciso adotar uma técnica de previsão estática
  para a primeira ocorrência de uma instrução de desvio ou de um padrão de desvios. No caso do processador UltraSPARC I, que utiliza uma previsão dinâmica por
  meio de um contador de dois bits, é possível assumir estaticamente que na primeira ocorrência de uma instrução de desvio o salto é tomado (contador
  iniciado em 10) ou não tomado (contador iniciado em 01) e, a partir do resultado da primeira ocorrência, o valor do contador é alterado dinamicamente conforme as instruções são
  repetidas.
  
  \nocite{ultrasparc1}
  \nocite{loadstore}
  \bibliographystyle{sbc}
  \bibliography{referencias}
\end{document}
